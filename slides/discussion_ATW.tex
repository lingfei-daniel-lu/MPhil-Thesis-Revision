\documentclass[10pt]{beamer}

\usepackage{amssymb,amsthm}% http://ctan.org/pkg/amssymb
\newtheorem{proposition}{Proposition}
\setbeamertemplate{theorems}[numbered]
\usecolortheme[RGB={100,0,0}]{structure}
\setbeamercolor{block title}{use=structure,fg=white,bg=structure.fg!75!black}
\setbeamercolor{block body}{parent=normal text,use=block title,bg=block title.bg!10!bg}

\setbeamertemplate{footline}[frame number]

\usepackage{amsmath}
\DeclareMathOperator*{\argmax}{argmax}
\DeclareMathOperator*{\argmin}{argmin}

\usepackage{natbib}
\bibliographystyle{aea}

\usepackage{xcolor}
\usepackage{graphicx}
\usepackage{amsmath}
\usepackage{numprint}
\npdecimalsign{.}
\nprounddigits{3}
\usepackage{colortbl}
\usepackage{appendixnumberbeamer}
\usepackage{subfigure}
\usepackage{comment}

\usepackage{hyperref}
\usepackage{caption}
\setbeamerfont{caption}{size=\small}
\usetheme{Singapore}
\usecolortheme{beaver}

% for adding regression tables
\usepackage{dcolumn}
% column to line up decimals
\usepackage{booktabs,caption}
\captionsetup[table]{name=Table}
\setlength{\abovecaptionskip}{-1pt}
\setlength{\belowcaptionskip}{-1.5pt}
\usepackage[flushleft]{threeparttable} 
% The above two allow that last line with the dagger as a bottom note.

\usepackage{xcolor}
\definecolor{underbrace}{RGB}{30,199,166}
\newcommand{\textfrac}[1]{
  \begin{tabular}{@{}l@{}}#1\end{tabular}
}

\title[ERPT]{Discussion on Trade Liberalization, Export and Product Innovation}

\author{Author: Prof. Sizhong SUN \\ \vspace{2mm} Discussant: Lingfei LU}

\date{Australasian Trade Workshop (ATW) \\ \vspace{2mm} Christchurch, New Zealand \\ \vspace{4mm} March 17, 2024 }

\begin{document}
	
\begin{frame}
    \maketitle
    \centering
\end{frame}

\begin{frame}{Summary}
    \begin{itemize}
	\item This paper studies how trade cost reduction will promote firms' exports and product innovation.
        \item It uses a four-step estimation algorithm to model all decisions of a firm: entry, exit, export, and innovation.
        \item It proposes and quantitatively verifies one static and two dynamic channels of the above effects:
        \begin{itemize}
            \item Static 1: The marginal benefit of trade cost reduction is larger for innovating-exporting firms than non-innovating-exporting firms.
            \item Dynamic 1: Export promotes firms' transition to a better state.
            \item Dynamic 2: Current innovation saves costs of innovation in the future.
        \end{itemize}
    \end{itemize}
\end{frame}

\begin{frame}{Comments and Questions}
Overall, I enjoy reading this innovative paper!
    \begin{itemize}
        \item This paper constructs a useful analytical framework connecting firms' exports and innovation, which could be extended to related topics.
        \item In each step of the firm decision, firms have perfect information about all shocks in future steps without strategic interaction. Strictly speaking, this is not a "game".
        \item The identification of sunk entry costs of export and innovation could be further validated by firms' whole history in addition to the previous period.
        \item In counterfactual analysis, are there other reasons to explain why simulated probabilities of export and product innovation are all higher than non-parametrically computed ones?
        \item (Minor) It may be better to introduce the meaning of the parameters in the model when they first appear in any formula.
        \begin{itemize}
            \item For example, what does $\psi$ explicitly mean? How do we understand the evolution of state variable $(\frac{\psi}{{w+m}})$?
        \end{itemize}
    \end{itemize}
\end{frame}

\end{document}