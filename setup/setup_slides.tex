\usepackage{amssymb,amsthm}% http://ctan.org/pkg/amssymb
\newtheorem{proposition}{Proposition}
\setbeamertemplate{theorems}[numbered]
\usecolortheme[RGB={100,0,0}]{structure}
\setbeamercolor{block title}{use=structure,fg=white,bg=structure.fg!75!black}
\setbeamercolor{block body}{parent=normal text,use=block title,bg=block title.bg!10!bg}

\setbeamertemplate{footline}[frame number]

\usepackage{amsmath}
\DeclareMathOperator*{\argmax}{argmax}
\DeclareMathOperator*{\argmin}{argmin}

\usepackage{natbib}
\bibliographystyle{aea}

\usepackage{xcolor}
\usepackage{graphicx}
\usepackage{amsmath}
\usepackage{numprint}
\npdecimalsign{.}
\nprounddigits{3}
\usepackage{colortbl}
\usepackage{appendixnumberbeamer}
\usepackage{subfigure}
\usepackage{comment}

\usepackage{hyperref}
\usepackage{caption}
\setbeamerfont{caption}{size=\small}
\usetheme{Singapore}
\usecolortheme{beaver}

% for adding regression tables
\usepackage{dcolumn}
% column to line up decimals
\usepackage{booktabs,caption}
\captionsetup[table]{name=Table}
\setlength{\abovecaptionskip}{-1pt}
\setlength{\belowcaptionskip}{-1.5pt}
\usepackage[flushleft]{threeparttable} 
% The above two allow that last line with the dagger as a bottom note.

\usepackage{xcolor}
\definecolor{underbrace}{RGB}{30,199,166}
\newcommand{\textfrac}[1]{
  \begin{tabular}{@{}l@{}}#1\end{tabular}
}